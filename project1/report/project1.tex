\documentclass[11pt]{article} % use larger type; default would be 10pt

\usepackage{geometry} % to change the page dimensions
\geometry{a4paper} % or letterpaper (US) or a5paper or....

\usepackage{amsmath}


\title{Project 1}
\date{} 

\begin{document}
\maketitle

\section{The Physical Problem}

The parachuter experiences the following forces :
\begin{itemize}
\item Gravity, $F_g$
\item Drag, $F_d$
\item Buouyancy, $F_b$
\end{itemize}
Newton's second law states that the sum of forces acting on the body equals the mass of the body times acceleration :

\[ ma = \sum F_{ext}\]
or in the case of the parachuter

\[ m\frac{dv}{dt} = F_g + F_d + F_b\]
where

\begin{equation*}
F_g = mg, \hspace{2mm}
F_d = -\frac{1}{2}C_D\rho|v|v, \hspace{1mm} \text{and} \hspace{2mm}
F_b = -\rho gV.
\end{equation*}

\subsection{Numerical Scheme}
In general we can write the ODE as following :
\begin{align} 
\dot{v} = -a(t)|v|v + b(t) \label{eq:1} \tag{1.a}\\
     v(0) = v_0 \tag{1.b}
\end{align}
for any a(t) and b(t), and start velocity $v_0$. For our physical problem  a = $\frac{C_D \rho}{2m}$ 
and b = g - $\frac{\rho gV}{m}$.Using the approximation $v'(t_n) \approx \frac{v^{n+1} - v^n}
{t^{n+1} - t^n} $, the numerical scheme then takes the following form :\\
\begin{align}
v^{n+1} &= \frac{v^n + b^{n+\frac{1}{2}}\Delta t}{1 + \Delta t a^{n+\frac{1}{2} }|v^n|}  
\label{eq:2} \tag{2}\\
a^{n+\frac{1}{2}} &= \frac{a^{n+1}+a^n}{2} \notag\\
b^{n+\frac{1}{2}} &= \frac{b^{n+1}+b^n}{2} \notag
\end{align}
where $a^{n} = a(t_n) \text{ and } b^{n}  = b(t_n)$ are function values evaluated at the mesh points.

We have managed to simplify the scheme by using geometric averaging for the term $|v|v$. By 
doing so we have linearized the nonlinear ODE \eqref{eq:1}. Finally we employed the Crank 
Nicolson scheme by evaluating the node points at $n+\frac{1}{2}$. Hence forth we will refer to b as 
the source term when we perform our tests.

\section{Verification \& Debugging}
After we have implemented the solver, we should \emph{always} test it for simple cases where 
we can expect machine precision. A good exercise is to check for a constant solution. To accomplish 
this, we have to construct the right ODE problem. We can specify any a(t), then we can fit the source 
term to the problem :

For $v_e$ = 4\footnote{To satisfy this solution, the initial condition $v_e(0)$ must equal 4.},
\begin{align*}
0 &= -16a(t) + b(t) \\
&\Downarrow \\
b(t) &= 16a(t)
\end{align*}
Inserting this expression for b(t) into the scheme
\begin{align*}
v^{n+1} = \frac{v^n + 16\Delta t a^{n+\frac{1}{2}}}{1 + \Delta t a^{n+\frac{1}{2}}|v^n|}.
\end{align*}
If we now let $v^n = 4$, then the scheme becomes
\begin{align*}
v^{n+1} &= 4\frac{1 + 4\Delta t a^{n+\frac{1}{2}}}{1 + 4\Delta t a^{n+\frac{1}{2}}} \\
             &= 4.
\end{align*}
We realize that it may be a good idea to design a general solver which can handle any a(t) and b(t). 
That is the reason why we have chosen to do an averaging of the function values. 

In the exercise we are required to demonstrate that a linear funciton of \textbf{t} does not fullfill 
the discrete equation (because of the geometric mean used for the quadratic drag term). Let us
then observe what happens  if we try the solution $v_e = ct$. The discrete solution becomes $v^n 
= c\Delta t n$. This implies that we should expect $v^{n+1} = c\Delta t (n+1)$. Assume that the 
source term, b, is zero. Then the scheme \eqref{eq:2} can be written as following

\begin{align*}
v^{n+1} = \frac{c\Delta t n}{1 + \Delta t a^{n+\frac{1}{2}}|c\Delta t n|}.
\end{align*}
This is obviously not $c\Delta t (n+1)$ \footnote[1]{Note that c is simply a constant.} . We can fix this 
problem by choosing b(t) such that it fits the scheme. As we did for the linear case, if we 
perform the same calculations as above, we end up with the desired result :

\begin{align*}
c &= -a(t)ct|ct| + b(t) \\
&\Downarrow \text{Rearranging for b(t)}\\
b(t) &= c + a(t)ct|ct| \\
&\Downarrow \text{The discrete version, after using geometric averaging}\\
b^{n+\frac{1}{2}} &= c + a^{n+\frac{1}{2}}c\Delta t(n+1)|c\Delta t n| \\
&\Downarrow  \text{Inserting this expression in our scheme}\\
v^{n+1} &= \frac{c\Delta t (n+1) + \Delta t a^{n+\frac{1}{2}}c\Delta t(n+1)|c\Delta t n|}{1 + 
\Delta t  a^{n+\frac{1}{2}}|c\Delta t n|}  \\
             &= c\Delta t (n+1)
\end{align*}
which is what we wanted. 
\end{document}
